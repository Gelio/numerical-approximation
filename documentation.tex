\documentclass[12pt]{article}
\usepackage{polski}
\usepackage[utf8]{inputenc}
\usepackage{amsfonts}
\usepackage{amsmath}
\usepackage{enumitem}
\usepackage{graphicx}
\usepackage{float}
\usepackage{centernot}
\setlength{\parskip}{1em}


\begin{document}
	\title{Sprawozdanie\\Metody Numeryczne 2, laboratorium 4}
	\author{Grzegorz Rozdzialik (D4, grupa lab. 2)}
	\maketitle	
	
	\section{Zadanie}
	{\Large Temat \textbf{4}, zadanie \textbf{46}:}\\
	Wzory empiryczne. Baza: $1$, $x$, $x^2$, $\sin x$. Graficzne przedstawienie punktów pomiarowych i funkcji przybliżającej.
	
	Niech $f: \mathbb{R} \to \mathbb{R}$ będzie funkcją aproksymowaną,
	$m \in \mathbb{R}$,
	$x_1, x_2, \dots, x_m \in \mathbb{R}$
	oraz $f_1, f_2, \dots, f_m \in \mathbb{R}$ będą wartościami funkcji aproksymowanej w punktach $x_i$ ($f_i = f(x_i), i = 1, 2, \dots, m$).
	
	Niech $g_1, g_2, \dots, g_n: \mathbb{R} \to \mathbb{R}$ będą funkcjami z bazy.
	
	Należy znaleźć element optymalny
	$$f^* = \alpha_1 g_1(x) + \alpha_2 g_2(x) + \dots + \alpha_n g_n(x)$$,
	który minimalizuje wyrażenie
	$$H(\alpha_1, \alpha_2, \dots, \alpha_n) = \sum_{i=1}^{m} \Big(f_i - f^*(x_i)\Big)^2$$
	Zakładamy $m \geq n$.
	
	Po obliczeniu pochodnej funkcji $H$ ze względu na dowolną zmienną i przyrównaniu jej do zera (ponieważ szukamy ekstremum) otrzymujemy następującą równość:
	\begin{equation}
		\forall_{k = 1, \dots, n} \hspace{10pt}
		\sum_{j=1}^{n} \alpha_j \sum_{i=1}^{m} g_j(x_i) g_k(x_i) = \sum_{i=1}^{m} f_i g_k(x_i)
		\label{rownosc-pochodnej-trudna}
	\end{equation}
	Zauważmy, że $\sum_{i=1}^{m} g_j(x_i) g_k(x_i) = \langle g_j, g_k \rangle$ oraz 
	$\sum_{i=1}^{m} f_i g_k(x_i) = \langle f_i, g_k \rangle$. Zatem równość (\ref{rownosc-pochodnej-trudna}) można zapisać jako:
	\begin{equation}
		\forall_{k = 1, \dots, n} \hspace{10pt}
		\sum_{j=1}^{n} \alpha_j \langle g_j, g_k \rangle = \langle f_i, g_k \rangle
		\label{rownosc-pochodnej-trudna}
	\end{equation}
	
	Otrzymaliśmy więc układ równań normalnych $G \alpha = F$, gdzie
	\begin{equation*}
		\alpha =
		\begin{bmatrix}
			\alpha_1 \\
			\alpha_2 \\
			\vdots   \\
			\alpha_n
		\end{bmatrix}
	\end{equation*}
	\begin{equation*}
		G =
		\begin{bmatrix}
			\langle g_1, g_1 \rangle & \langle g_1, g_2 \rangle & \dots  & \langle g_1, g_n \rangle \\
			\langle g_2, g_1 \rangle & \langle g_2, g_2 \rangle & \dots  & \langle g_2, g_n \rangle \\
			\vdots                   & \vdots                   & \ddots & \vdots                   \\
			\langle g_n, g_1 \rangle & \langle g_n, g_2 \rangle & \dots  & \langle g_n, g_n \rangle
		\end{bmatrix}
	\end{equation*}
	\begin{equation*}
		F =
		\begin{bmatrix}
			\langle f_1, g_1 \rangle \\
			\langle f_2, g_2 \rangle \\
			\vdots                   \\
			\langle f_n, g_n \rangle
		\end{bmatrix}
	\end{equation*}
	Po jego rozwiązaniu otrzymujemy współczynniki $\alpha_i$, a mamy całą postać elementu optymalnego $f^*$.



	W przypadku naszego zadania mamy:
	\begin{align*}
	g_1(x) &= 1\\
	g_2(x) &= x\\
	g_3(x) &= x^2\\
	g_4(x) &= \sin x
	\end{align*}
	Ilość punktów pomiarowych $n$ nie może być mniejsza niż 4 ($n \geq 4$). Wtedy element optymalny $f^*$ ma postać:
	$$
	f^*(x) = \alpha_1 + \alpha_2 x + \alpha_3 x^2 + \alpha_4 \sin x
	$$

	
	\section{Opis metody}
	\textbf{TODO}
	
	
	
	
	
	\section{Implementacja metody}
	\textbf{TODO}
	
	
	
	
	\section{Poprawność metody}
	\textbf{TODO}
	
	
	
	
	\section{Przykłady}
	\textbf{TODO}
	
	\begin{enumerate}[label=\textbf{Przykład \arabic*}]
		\item
		\textbf{TODO}
		
	\end{enumerate}
	
	
	
	
	
	
	\section{Wnioski}
	\begin{enumerate}
		\item \textbf{TODO}
	\end{enumerate}

	
	
	
	
	
	\section{Skrypt do testowania}
	\textbf{TODO}
	
	
	
	\section{Bibliografia}
	\begin{enumerate}
		\item Informacje z wykładu \textit{Metod numerycznych 2} (wydział MiNI PW, dr Iwona Wróbel) \textbf{TODO} (dodać więcej informacji)
	\end{enumerate}
	
\end{document}