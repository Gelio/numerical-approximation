\documentclass[12pt]{article}
\usepackage{polski}
\usepackage[utf8]{inputenc}
\usepackage{amsfonts}
\usepackage{amsmath}
\usepackage{enumitem}
\usepackage{graphicx}
\usepackage{float}
\usepackage{centernot}
\setlength{\parskip}{1em}


\begin{document}
	\title{Sprawozdanie\\Metody Numeryczne 2, laboratorium 4}
	\author{Grzegorz Rozdzialik (D4, grupa lab. 2)}
	\maketitle	
	
	\section{Zadanie}
	{\Large Temat \textbf{4}, zadanie \textbf{46}:}\\
	Wzory empiryczne. Baza: $1$, $x$, $x^2$, $\sin x$. Graficzne przedstawienie punktów pomiarowych i funkcji przybliżającej.

	
	\section{Opis metody}
	\textbf{TODO}
	
	
	
	
	
	\section{Implementacja metody}
	\textbf{TODO}
	
	
	
	
	\section{Poprawność metody}
	\textbf{TODO}
	
	
	
	
	\section{Przykłady}
	\textbf{TODO}
	
	\begin{enumerate}[label=\textbf{Przykład \arabic*}]
		\item
		\textbf{TODO}
		
	\end{enumerate}
	
	
	
	
	
	
	\section{Wnioski}
	\begin{enumerate}
		\item \textbf{TODO}
	\end{enumerate}

	
	
	
	
	
	\section{Skrypt do testowania}
	\textbf{TODO}
	
	
	
	\section{Bibliografia}
	\begin{enumerate}
		\item Informacje z wykładu \textit{Metod numerycznych 2} (wydział MiNI PW, dr Iwona Wróbel) \textbf{TODO} (dodać więcej informacji)
	\end{enumerate}
	
\end{document}